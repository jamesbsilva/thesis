% !TEX root = jbsilvaThesis.tex

%%%%%%%%%%%%%%%%%%%%%%%%%%%%%%%%%%%%%%%%%%%%%%%%%%%%%%%%%%%%%
%%%%%%%%%%%%%%%%%%%%%%%%%%%%%%%%%%%%%%%%%%%%%%%%%%%%%%%%%%%%%

\chapter{\label{chp:intro}Introduction}
In this dissertation work will be presented that explores the role of impurities in systems with \lr\ effective interactions. It will be shown that impurities and \lr\ interactions are crucial in modeling many real world systems such as metals, polymers, earthquake fault systems, and biological systems. In particular, two models will be explored: an earthquake fault models, and a \het\ Ising model. 

Initially, the properties of homogeneous nucleation in the Ising model will be reviewed to allow for the introduction of a \het\ Ising model. The effectiveness of percolation techniques in defining nucleation properties near the spinodal for the \het\ Ising model will be explored. The properties of droplets in this system with impurities will be discussed and compared to the pure Ising model.

In the modeling of earthquake fault systems the properties of two types of disorder will be studied. The role of asperities in an earthquake fault model will be explored to gain insight into the observed quasiperiodicity in real world earthquake fault systems and the empirical law known as Omori's law. Lastly, the role of man-made microfractures occurring during hydraulic fracturing that will be modeled by a modification of an established model for earthquake fault systems. The effect of these microfractures on the statistics of earthquake event sizes will be measured for different models of the effect of the microfractures on the earthquake fault system. The question of  ergodicity in these models will be explored to determine the level of generality of results for different realizations of induced microfractures. 

\section{Structure of Dissertation}
Chapter~\ref{chp:intro_nuc} will introduce the prerequisite concepts and background necessary to understand  problems in nucleation. A summary of relevant research that forms the context of this research  is also provided in the second chapter. The second chapter will also delineate some of the applications of this research that motivate the value of studying nucleation. 

Chapter~\ref{chp:nuc_details} will dive deeper into the details of the techniques used to study nucleation such as classical nucleation theory and percolation methods for spinodal nucleation. The goal of this discussion will be to provide a deeper understanding of previous work done on nucleation and provide references to explore other approaches to  nucleation. This chapter will conclude with research work aimed at understanding \textit{effective} interactions by determining how the nucleation energy barrier changes for different quench levels.

Chapters~\ref{chp:pseudo} and \ref{chp:het_nucl} will cover work on \het\ nucleation in Ising systems. The first of these chapters will discuss work defining the spinodal for a \het\ Ising system. The fifth chapter will focus on the changes to the structure of the saddle point object defining the nucleating droplet which initiates the decay of the metastable state.

Chapters~\ref{chp:intro_quakes}--\ref{chp:asper} focus on the physics of \het\ systems with \lr\ interactions in earthquake fault systems. The first of these chapters  summarizes previous research on earthquake fault systems and introduce work which connects two models: the \ofc\ model, and the \rjb\ model. Chapter~\ref{chp:asper} will focus on improving models for earthquake fault systems by introducing the effects of having varying hardness of the faults in the earthquake fault system. The consequences of this improvement will be illustrated through work on the statistics of aftershocks in this model.

Work on modeling an earthquake fault system undergoing hydraulic fracturing will be discussed in Chapter~\ref{chp:frack}. This discussion will start with an introduction  to hydraulic fracturing.  The oil-water interface will be modeled using invasion percolation. The model for hydraulic fracturing will then be constructed by combining elements from the individual physical processes. The statistics for the events  will be determined. This will lead to a discussion of the applicability of the results for an individual fault to other faults. This question will be approached by the study of ergodicity in the hydraulic fracturing earthquake fault model.
 
The concluding chapter will review the results discussed in this thesis. Future work will conclude the discussion of \het\ systems, nucleation, and earthquake fault systems.

