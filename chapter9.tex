% !TEX root = jbsilvaThesis.tex

\chapter{\label{chp:conclusion}Conclusions}

\section{Heterogeneous Nucleation}

In this thesis the role of quenched disorder in the Ising model near the spinodal has been studied and shown to retain some of the properties of spinodal nucleation without disorder, assuming some appropriate corrections are made to the theory. These corrections are motivated by the reformulation of the \het\ Ising model as a diluted random field Ising model. The correction to the field in spinodal nucleation theory was derived for fixed spins in the same direction. The dilution correction was obtained from previous work by Liu et al.~\cite{kangdilute}. It was shown that in the long-range interaction limit and small fixed spin density this correction correctly describes the percolation clusters in spinodal nucleation theory. 

With an understanding of the modification of the spinodal field in the \het\ Ising model the nature of the saddle point object describing the nucleating droplet was explored. Measurements of the stable spin density interface were obtained and it was found  that the \het\ system produces a sparser droplet when compared to the \homo\ case. This sparseness was greater than the expected value from the correction to the spinodal field  due to the effective field effect that  fixed spins introduce to the system.
 
In the process of studying the \het\ Ising model it was determined that the intervention method could be reinterpreted to understand the nature of interactions in a system where complex interactions may obfuscate the nature of the nucleation process. The work on nucleation opens the possibility to understanding more complicated systems and interactions as will be discussed in Sec.~\ref{sec:future}.

\subsection{\label{sec:future}Further work}

Many systems have complex interactions that obscure the type of nucleation present. By applying the method in Section~\ref{nonmonoind} to these systems, it is possible to determine the type of nucleation occurring in these systems. Clathrate hydrates are an example of a system that would benefit from such work. 
 
Extending the work on the limits of metastability in \het\ Ising models to obtain a version of Harris criterion applicable to quenched systems would be an area of interest. Figure~\ref{fig:spinodalshift}  provides a good starting point for such work. A Harris-like criterion for quenched magnetic impurities is of interest because it is clear from the work of others in \het\ Ising systems~\cite{poole13} that large numbers of  fixed spins can immediately induce nucleation and greatly affect the nucleation barrier. The  understanding of how these different realizations of distributions affect the  sparseness observed in the nucleating droplets is a natural extension of this work for larger fixed spin densities. Of particular interest is work on the nucleation rate in \het\ \lr\ models. Preliminary work was done to confirm the nucleation rate increases as the number of fixed spins is increased, but a theory to collapse the metastable lifetime curves has not been completed.


\section{Understanding Disorder in Earthquake Fault Systems}

A model of earthquake faults was introduced and a connection mapping the \ofc\ model to another earthquake fault model, the \rjb\ model, was done. This mapping allowed the focus of the work to be on the \ofc\ model. The \ofc\ model was  extended to include the role of asperities. It was shown that by introducing  asperities into the \ofc\ model, the system appeared to not only maintain the event size statistics relation known as Gutenberg-Richter scaling, but also displayed the aftershock statistics behavior known as Omori's law. The importance of  a model that maintains Gutenberg-Richter scaling and Omori's law is emphasized by the work of Sornette~\cite{sornette03}, which shows that these empirical laws lead to other effects such as Bath's law. The introduction of heterogeneity in the form of asperities was shown to be crucial in understanding a possible physical mechanism for the empirically observed behavior.

The \ofc\ model is further extended in the context of hydraulic fracturing by coupling the model to the invasion percolation model for an oil-water interface. It was shown that the excess of larger events is unique to the fault due to the heterogeneity breaking the ergodic property of an effectively ergodic \ofc\ system. It was  observed that  hydraulic fracturing increases the size of of middle-sized events at the expense of the very large events, especially in the very beginning and end of the hydraulic fracturing process. 

\subsection{Further work}

This work shows the interesting properties of asperities in the behavior of aftershocks. One of the goals of earthquake fault systems research is  understanding predictive information from the possible measurements. An interesting question in this  direction is distinguishing the statistics of large events due to  an asperity from a stochastically generated large event. Is there an event large enough to create a memory in the  system for an earthquake fault system similar to the behavior induced by an asperity? It was determined that the memory created by an asperity slip event lingers for thousands of plate updates. Aside from the dissipation time it is also of interest to look at the spatial properties of the dissipation of this memory left after an asperity slip event. 

Generating stress failure thresholds for many asperity sites from a Gaussian distribution was observed to introduce a damped sinusoidal to the typical Omori's law behavior. This work can be extended by drawing the stress failure threshold from two Gaussian distributions with a large amount of separation of their means  to determine if  the resulting stress ``memory" distribution is the result of a simple superposition. If such a superposition of the stress ``memory" distribution occurs, it would motivate the use of frequency analysis of the large event rate after removing the Omori's law term in Eqn.~\eqref{eq:omorismod} to possibly determine information of the asperities in a fault system. It is also of interest to draw the asperity stress failure threshold from a distribution without a well defined mean such as a power law.


The work in this thesis has shown that some of the properties of the \ofc\ model without asperities are maintained in the system with asperities. However, it remains to be seen if the introduction of asperities changes the equilibrium properties such as the Boltzmann distribution of the energy defined in Eqn.~\eqref{eq:ofcenergy}. 

In my work on asperities these parameters where fixed in the regime of a clearly equilibrium system as observed in Fig.~\ref{fig:ofcmetricsumsq}. The effect of relaxing this equilibrium behavior on the observed Omori's law has not been explored. The guidance obtained from connecting the \ofc\ model to the \rjb\ model would allows one to know the possible range of the dissipation and residual stress noise parameters. 

Of interest in the work on hydraulic fracturing modeling is determining the distribution of the large events while the hydraulic process is occurring. This would allow one to determine if the excess large earthquake events  occur right after an update in the hydraulic fracturing process or if these excess events are uniformly distributed.


%%%%%%%%%%%%%%%%%%%%%%%%%%%%%%%%%%%%%%%%%%%%%%%%%%%